%% SECTION 1
%\chapter{Executive summary}
%\label{chap:ExecutiveSummary}

The goal of the {\em Neutrino Experiment with a Xenon TPC} ({\bf NEXT})\footnote{\url{http://next.ific.uv.es/next}} is the construction, commissioning and operation of the NEXT-100 detector, a high-pressure, xenon (HPXe) Time Projection Chamber using electroluminescent (EL) readout. NEXT-100 will search for neutrinoless double beta decay  (\bbonu) events in \XE, deploying 100 kg of xenon enriched at 90\% in the isotope \XE. The host of the experiment is the Canfranc Underground Laboratory (LSC). 

The project is being developed in three phases. The initial R\&D extended from 2010 to 2014. The operation of three large (1-kg xenon at 10 bar pressure) prototypes, 
NEXT-DBDM (LBNL), NEXT-DEMO (IFIC) and NEXT-MM (Zaragoza), has shown the excellent performance (energy resolution, electron reconstruction) of the chosen EL technology, as well as explored alternative options such as operation with gain (using Micromegas readout) in xenon-TMA mixtures. 
Some of the most important results from this period include:

\begin{enumerate}
\item The design and characterisation of the SiPM tracking system \cite{Alvarez:2012haa}.
\item PMT calibration in situ procedures \cite{Freitas:2015tha}.
\item The procedures and results of the radiopurity campaign for NEXT-100 \cite{Alvarez:2012as, Alvarez:2014kvs, Cebrian:2015jna}.
\item Description of the trigger system for NEXT-DEMO (and its extrapolation to NEXT-100)
\cite{Esteve:2012hy}.
\item Description of the front-end electronics for energy measurements \cite{Gil:2012sr}.  
\item Performance of EL prototypes\cite{Alvarez:2012hh, Alvarez:2013gxa, Alvarez:2012hu,Lorca:2014sra}.   
\item Initial results on operation with Xe-TMA mixtures \cite{Alvarez:2013kqa, Alvarez:2013oha}.
\item Measurements with alpha particles and nuclear recoils \cite{Renner:2014mha, Serra:2014zda}.
\item Topological signature \cite{Ferrario:2015kta}.
\item Physics potential of the detector \cite{GomezCadenas:2012jv, MartinAlbo:2013ve,Gomez-Cadenas:2013lta, Martin-Albo:2015rhw}.              
\end{enumerate}

The second phase of the project is the construction, commissioning and operation of NEW (NEXT-WHITE)\footnote{The name honours the memory of the late Professor James White, one of the key scientists of the NEXT Collaboration.}, a first-stage, radiopure, 10-kg demonstrator intended to exercise the NEXT-100 detector technical solutions and infrastructures (including the gas system and the Slow Controls), as well as to provide essential data for the NEXT background model. The apparatus and the infrastructures have been built during 2015 are currently being commissioned at the LSC. The foreseen operation period is two years (2017 and 2018). 

The third stage of the project is the construction, commissioning and operation of NEXT-100. The collaboration intends to present an updated Technical Design Report (TDR) to the LSC Scientific Committee in November-2017, seeking for approval to start construction of NEXT-100. The assembly of the NEXT-100 detector is foreseen to occur during 2018, and commissioning is expected in 2019. If the background model expectations are confirmed, NEXT-100 could improve the sensitivity of both EXO-100 and KamLAND-Zen in 2--3 years of data taking\cite{Martin-Albo:2015rhw}. 

In this report we present an overview of the  project as of May 2016, stressing the achievements during 2015. The report is organised as follows. Section \ref{sec.infra} describe the infrastructures needed at the LSC for the operation of the NEW and NEXT-100 detectors.  Most of those infrastructures have been built or completed during 2015. Section \ref{sec.new} presents the status of the NEW apparatus, which has also been assembled during 2015. Conclusions are presented in section \ref{sec.conclu}.



%Each part of this report is summarised in an executive summary chapter.

