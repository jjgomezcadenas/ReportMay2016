\section{Conclusions and outlook}
The front-end cabling and electronics requirements for NEW and NEXT-100 pose harder design constraints than those in NEXT-DEMO. Lessons learned with the front-end in NEXT-DEMO allowed the new solutions presented in this paper.

A complete slice of the NEW tracking plane readout has been tested. It consists of a KDB with SiPMs from SensL and Hamamamtsu, 4 m of cabling, a FEB64 front-end card, a DAQ interface module (FECv3 from RD51's SRS) and a DAQ PC running the DAQ software. 
Measurements based on dark count events and also with a LED source show good performance for the NEXT experiment and a correct behaviour (and so a validation) 
of the full readout chain. Nevertheless, the front-end board had minor design revision in late 2014 to correct some mechanical aspects as well as to reduce the noise produced by the on-board DC/DC converter that powers the FPGA core. The test of the new boards is being done right now.

The full characteristics comparison between the SiPMs from SensL and Hamamatsu showed clearly that SensL SiPMs performance fits better our requirements, and also radiopurity tests indicates that Hamamatsu SiPMs will increase dramatically our detector background. For these reasons we decides to use the SiPMs from SensL for the next steps of NEXT Experiment.

The construction and operation of NEW in 2015 will be a challenge, though good tracking results obtained with NEXT-DEMO and the measured performance of the NEW readout slice give cause for optimism. Beyond NEXT, the present work shows the feasibility of tracking applications based on SiPMs and discrete electronics in medium-scale detectors.